% !TEX bibfile = ref.bib
\documentclass[11pt]{article}
\usepackage{latexsym}
\usepackage{amsmath,amssymb,amsthm}
\usepackage{epsfig}
\usepackage[right=0.8in, top=1in, bottom=1.2in, left=0.8in]{geometry}
\usepackage{setspace}
\usepackage{algorithm}
\usepackage{algorithmic}
\renewcommand{\algorithmicrequire}{\textbf{Input:}}  % Use Input in the format of Algorithm
\renewcommand{\algorithmicensure}{\textbf{Output:}} % Use Output in the format of Algorithm
\usepackage{listings}
\usepackage{ctex}

% 用来设置附录中代码的样式

\lstset{
    basicstyle          =   \sffamily,          % 基本代码风格
    keywordstyle        =   \bfseries,          % 关键字风格
    commentstyle        =   \rmfamily\itshape,  % 注释的风格,斜体
    stringstyle         =   \ttfamily,  % 字符串风格
    flexiblecolumns,                % 别问为什么,加上这个
    numbers             =   left,   % 行号的位置在左边
    showspaces          =   false,  % 是否显示空格,显示了有点乱,所以不现实了
    numberstyle         =   \zihao{-5}\ttfamily,    % 行号的样式,小五号,tt等宽字体
    showstringspaces    =   false,
    captionpos          =   t,      % 这段代码的名字所呈现的位置,t指的是top上面
    frame               =   lrtb,   % 显示边框
}

\lstdefinestyle{Python}{
    language        =   Python, % 语言选Python
    basicstyle      =   \zihao{-5}\ttfamily,
    numberstyle     =   \zihao{-5}\ttfamily,
    keywordstyle    =   \color{blue},
    keywordstyle    =   [2] \color{teal},
    stringstyle     =   \color{magenta},
    commentstyle    =   \color{red}\ttfamily,
    breaklines      =   true,   % 自动换行,建议不要写太长的行
    columns         =   fixed,  % 如果不加这一句,字间距就不固定,很丑,必须加
    basewidth       =   0.5em,
}

\spacing{1.06}

\newcommand{\handout}[5]{
  \noindent
  \begin{center}
  \framebox{
    \vbox{\vspace{0.25cm}
      \hbox to 5.78in { {SE3352:\hspace{0.12cm}Algorithm Design} \hfill #2 }
      \vspace{0.48cm}
      \hbox to 5.78in { {\Large \hfill #5  \hfill} }
      \vspace{0.42cm}
      \hbox to 5.78in { {#3 \hfill #4} }\vspace{0.25cm}
    }
  }
  \end{center}
  \vspace*{4mm}
}
\newcommand{\lecture}[4]{\handout{#1}{#2}{#3}{Scribes:\hspace{0.08cm}#4}{Notes #1}}

\newtheorem{theorem}{Theorem}
\newtheorem{corollary}[theorem]{Corollary}
\newtheorem{lemma}[theorem]{Lemma}
\newtheorem{observation}[theorem]{Observation}
\newtheorem{example}[theorem]{Example}
\newtheorem{definition}[theorem]{Definition}
\newtheorem{claim}[theorem]{Claim}
\newtheorem{fact}[theorem]{Fact}
\newtheorem{assumption}[theorem]{Assumption}
\newcommand{\E}{\textbf{E}}
\newcommand{\var}{\text{var}}
\def\eps{\ensuremath\epsilon}
\begin{document}

\lecture{1 -- Clustering}{Nov 27, 2021}{Instructor:\hspace{0.08cm}\emph{Guoqiang Li}}{\emph{Haotian Wang, Yiyan Li}}


\section{Introduction}
Clustering can be considered the most important unsupervised learning problem; As every other problem of this kind, it deals with finding a structure in a collection of unlabeled data.
\begin{definition}
A cluster is a collection of objects which are “similar” between them and are “dissimilar” to the objects belonging to other clusters. Clustering is the algorithm that recognizes clusters from a given data set.
\end{definition}
Note that this is a very rough definition. Part of common application domains in which the clustering problem arises are as follows:
\begin{itemize}
\item \textbf{Multimedia Data Analysis:} Learning image or video representations without manual annotations. e.g. When using Streaming media platform, face clustering can recognize all the actors in any frame. \cite{wang2019linkage,xing2021learning}
\item \textbf{Responding to public health crises:} With the increasing number of samples, the manual clustering of COVID-19 data samples becomes time-consuming. Clustering helps classify medical datasets deterministically.\cite{COVID-19}
\item \textbf{Social Network Analysis:} Clustering provides an important understanding of the community structure in the network. Results can be used for customer segmentation and sending ads. Put it in a formal way, \textbf{clustering groups the nodes of the graph into clusters}, taking into account the edge structure of the graph in such a way that there are several edges within each cluster and very few between clusters. \cite{community}
\item \textbf{Intermediate Step for other fundamental data mining problems:} Clustering can be considered as a form of data summarization. Many clustering methods are closely related to dimensionality reduction methods. Such methods can be considered a form of data summarization.
\item \textbf{Intelligent Transportation:} Under \textbf{the online scenario}, data is in the form of streams, i.e., the whole dataset could not be accessed at the same time. In future intelligent transportation, low-latency online vehicle tracking is essential and can be solved by online clustering.\cite{luna2021online}

\end{itemize}

\par Today we'll start from the naive K-means clustering and improve the algorithm step by step. The lecture has X main topics that we'll go through, i.e. TODO AL LAST!!

\section{Problem Description}
\par The k-means clustering problem is one of the oldest and most important questions in all of computational geometry. Given an integer $k$ and a set of $n$ data points in $\mathbb{R}^{d}$, the goal of this problem is to choose $k$ centers so as to minimize the total squared distance between each point and its closest center.\cite{kmeanspp}
\par There are several kinds of k-means algorithms among which the most common algorithm , also called naive k-means algorithm, was first proposed by Stuart Lloyd\cite{Least-squares-quantization-in-PCM} of Bell Labs in 1957.
\par For a k-means problem, we are given an integer $k$ and a set of data vector ($x_1, x_2, x_3 \dots x_n$) in $d$-dimension. And we need to choose $k$ centroids to partition the $n$ vectors into $k$ types $T$ ($T_1, T_2 \dots T_k$) with the minimum within-cluster sum of squares ($WCSS$)
$$\mathcal{WCSS} = \arg \min \sum_{i=1}^{k} \sum_{x \in S_i } {\left\lVert x - c _i \right\rVert }^2  $$
\par where $\mu_i$ is the mean of vector in set $S_i$.

\begin{lemma}
  Let $S$ be a set of points with its mean to be $\mu$, and let $c$ to be and arbitrary point.Then $\sum_{x \in S}{\left\lVert x - c\right\rVert}^2 = \sum_{x \in S}{\left\lVert x - \mu\right\rVert}^2 + \left\lvert S \right\rvert \cdot {\left\lVert z - \mu\right\rVert}^2 $
\end{lemma}
\par So we minimize the function only when $c_i=\mu_i$
\begin{equation*}
  \begin{split}
    \mathcal{WCSS} & = \arg \min \sum_{i=1}^{k} \sum_{x \in S_i } {\left\lVert x - c _i \right\rVert }^2 \\
    & = \arg \min \sum_{i=1}^{k} (\sum_{x \in S_i} {\left\lVert x - \mu_i\right\rVert}^2 + \left\lvert S_i \right\rvert \cdot {\left\lVert c_i - \mu_i\right\rVert}^2) \\
    & = \arg \min \sum_{i=1}^{k} {\left\lvert S_i \right\rvert }\cdot Var S_i \\
    & = \arg \min \sum_{i=1}^{k} \frac{1}{2\cdot\left\lvert S_i\right\rvert } \sum_{x,y \in S_i} {\left\lVert x_i - y_i\right\rVert }^2
  \end{split}
\end{equation*}

\section{Algorithms}
\subsection{The K-means algorithm}
The k-means algorithm is a simple and fast algorithm for this problem, although it offers no approximation guarantees at all.
It iteratively calculates the sum of distance within a cluster and updates the partition.The details are as follows.\cite{k-means}
\begin{enumerate}
  \item Arbitrarily choose and initial $k$ centers $\mathcal{C} = \{c_1, c_2 \dots c_k\}$
  \item For each $i \in \{1, 2 \dots k\}$, set the cluster $C_i$ to be the set of points that are closer to $c_i$ than they are to $c_j$ for all $j \neq i$
  \item For each $i \in \{1, 2 \dots k\}$, set $c_i$ to be the center of all points in $C_i$ 
  \item Repeat Step 2 and Step 3 until $\mathcal{C}$ no longer changes.
\end{enumerate}
\begin{algorithm}
  \caption{K-means}
  \label{k-means}
  \begin{algorithmic}
    \REQUIRE {k: number of output cluster; Data: input data}
    \ENSURE {$\mathcal{S}$ : set of all clusters $S_i$}
    \STATE Arbitrarily initialize k centroids $C=\{c_1, c_2 \dots c_k\}$ 
    \REPEAT
      \FOR {each point $x$ in Data $S$}
      \FOR {$i = 0 \rightarrow k$}
      \FOR {$j = 0 \rightarrow k$}
      \STATE {set x to be a member of cluster $S_i$ where ${\left\lVert x-c_i\right\rVert }^2 < {\left\lVert x-c_j\right\rVert }^2$}
      \ENDFOR
      \ENDFOR
      \ENDFOR
      \FOR {$i = 0 \rightarrow k$}
      \STATE {$c_i \leftarrow \frac{1}{\left\lvert S_i\right\rvert } \sum_{x \in S_i} x$}
      \STATE {set $c_i$ to be the centroid of all points in  cluster $S_i$}
      \ENDFOR
    \UNTIL{$S$ stays unchanaged}
    \ENSURE {$\mathcal{S}$ : set of all clusters $S_i$}
  \end{algorithmic}
\end{algorithm}

\subsection{Reduce cost of single iteration}
In both K-means and K-means$++$ algorithm, if there are $n$ data points in $\mathbb{R}^ d$ space and $k$ clusters for partition, each iteration involves $n * k$ distance computations, which would significantly slow down algorithm. One contribution from Siddhesh Khandelwal\cite{Faster} helps reduce this cost to $n * k^{\prime}$ ($k^{\prime} << k$) by generating candidate cluster list (CCL) of size $k^{\prime}$ for each data point. We'll show how this heuristic works in detail.
\subsubsection{Inspiration}
The optimization is based on the observation that across all iterations of K-means or K-means$++$, a data point changes its membership only among a small subset of clusters. The heuristic considers only a subset of \textbf{nearby cluster as candidates} for deciding membership for a data point. This heuristic has advantage of speeding up K-means and K-means$++$ clustering with marginal increase in loss functuion(MSE in our case). Note that the optimization can be applied in any algorithm to solve K-means problem with steps to caculate distance between points and centers, acting as augmentation.
\subsubsection{Augmentation target}
Let $A$ be any variant algorithm of K-means problem and $B$ be the same variant augmented with this geuristic. Let $T$ be the time required for $A$ to converge to MSE value of $E$. Let $T^{\prime}$ be the time required for $B$ to converge to MSE value of $E^{\prime}$. Our target is:
\begin{itemize}
\item \textbf{For convergence time:} $T^{\prime} < T$
\item \textbf{For loss:} $E^{\prime} \le E \space $ or $ \space E^{\prime} \overset{marginally}{>} E$
\end{itemize}

\subsubsection{Augmentation detail}
\par We assume that $k^{\prime}$ is significantly smaller than $k$. We will show how to choose $k^{\prime}$ later.We build candidate cluster list (CCL) based on top $k^{\prime}$ nearest clusters to the data point after first iteration of K-means. \par
\begin{definition}
\textbf{Cluster centroid} is the middle of a cluster. A centroid is a vector that contains one number for each variable, where each number is the mean of a variable for the observations in that cluster. The centroid can be thought of as the multi-dimensional average of the cluster.
\end{definition}
Consider a data point p1 and cluster centroids represented as $c_1$, $c_2$..., $c_k$. We assume that $k^{\prime}$ = 4, and $k^{\prime} << k $. After first iteration of K-means $c_5$, $c_6$, $c_8$, and $c_{11}$ are the top four closest centroids to $p_1$ in the increasing order of distance. This is the candidate cluster list for $p_1$. If we run K-means for second iteration, $p_1$ will compute distance to all k centroids. After second iteration, there are two possible cases:
\begin{enumerate}
\item The list do not change but only members'ranking changes.
\item Several members of the centroids in the previous list are replaced with other centroids which were not in the list.
\end{enumerate}
It seems that the augmentation makes no sense. But what makes this method succeed is \textbf{real world data rarely makes case 2 happen}. That is, the set of top few closest centroids for a data point \textbf{remains almost unchanged even though order among them might change}.
\subsubsection{Augmentation analysis}
Overhead analysis:
\begin{itemize}
\item \textbf{Computation overhead:} $O(nklog(k))$ for creating CCL at first. We have to compute the distance to each cluster's centroid for each point and sort them to create CCL. 
\item \textbf{Memory overhead:}$ O(nk^{\prime})$ to maintain CCL.
\end{itemize}
\par
\begin{proof}
Let $x_1, x_2 \dots x_n$ be n vectors in $\mathbb{R}^{d}$, then $f(x) = \sum_{i=1}^{n} {\left\lVert x_i - x\right\rVert}^2 $ gets its minimum iff. $x = \frac{1}{n} \sum_{i=1}^{n}x_i$
\begin{equation*}
  \begin{split}
    \frac{d f(x)}{d x} &=  \frac{d \sum_{i=1}^{n} {\left\lVert x_i - x\right\rVert}^2}{d x} \\
    & = -2\sum_{i=1}^{n} (x_i - x) \\
    & = 0 \\
    x & = \frac{1}{n} \sum_{i=1}^{n}x_i
  \end{split}
\end{equation*}

\par $x = \frac{1}{n} \sum_{i=1}^{n}x_i$ is a stationary point of this function. Owing that it is a strictly convex function, the stationary point is alse the only minimun point of that function. 
So the function gets its minimum at $x = \frac{1}{n} \sum_{i=1}^{n}x_i$.
\end{proof}
\begin{proof}
  Updated value $f(x'')$ is strictly less than the original $f(x')$ where  $x'' = \frac{1}{n} \sum_{i=1}^{n}x_i$.
\par As described above, each centroid is updated to the center of all points in cluster $C_i$. 
That is to say, once the centroid of one cluster changed from $x'$ to  $x'' = \frac{1}{n} \sum_{i=1}^{n}x_i$, the function gets its minimum in this iteration at $x''$ and $f(x'')<f(x')$.
\end{proof}
\subsection{The K-means++ algorithm}
The naive K-means algorithm do great work for the simplicity and efficiency, but it also has drawbacks. In the caes of initializing $k$ centroids using naive K-means algorithm (usually Lloyd's algorithm), we use randomization.
The initial $k$ centroid are picked in the range of data set randomly. However, this initialization strategy could result in initialization sensitivity. The final formed clusters could be affected greatly by the initail picked centroids.
\par Here are a few figures showing the potential:

\section{Key properties}
K means problem is an NP Hard problem, and two teams have proved them using 3-SAT and Exact Cover by 3-Sets respectively.\cite{npc-prove,Vattani_thehardness} Next, I'll try to describe the reduction from 3-SAT to k-means since NP-Complete problem is an inescapable topic in algorithm course and try my best to get rid of copying original material.
\subsection{Reduction from 3-SAT to K-means}
Let {F} be the given planar 3-SAT instance with n variables and m clauses. We construct an instance I of planar k-means corresponding to F. Properties of layout I are listed below:
\begin{enumerate}
\item Each variable $x_i$ corresponds to a simple circuit $s_i$ in the plane and each circuit has an even number Q of vertices. Each vertex on such a circuit have M copies of a point. Note that M and Q will be stricted below. Now we can partiution
\item b
\item cc
\end{enumerate}


\section{Median Trick}
So far, we have an algorithm $A$ which estimates in correct range of $\eps$ with probability $\ge 0.9$. Our new algorithm $A^{\ast}$ will output in range of $\eps$ with probability $1-\delta$.
Algorithm:
\begin{itemize}
\item Repeat $A$ for $m=O(log (1/\delta))$ times
\item Take median of all the $m$ answers.
\end{itemize}

To prove the correctness, we'll use Chernoff/Hoeffding bounds.

\begin{definition}
[Chernoff/Hoeffding Bound]
Let $X_{1}$, $X_{2}$, $\ldots$, $X_{m}$ be independent random variables $\in \{0,1\}$,
$\mu = E[\Sigma_{i} X_{i}], \eps \in [0,1]$.
Then $Pr[|\Sigma_{i} X_{i}-\mu| > \eps\mu] \leq 2e^{-\eps^{2}\mu/3}$
\end{definition}

Define $X_{i} = 1$ iff the $i^{th}$ answer of $A$ is correct (i.e. estimated value of $A$ lies in correct range).

\begin{claim}
$E[X_{i}] = 0.9$, and $E[\mu] = 0.9m$
\end{claim}

\begin{proof}
Since A is correct with probability 0.9, $E[X_{i}] = 0.9$. And $E[\mu] = 0.9m$ due to linearity of expectation.
\end{proof}

\begin{claim}
New algorithm $A^{\ast}$ is correct when $\Sigma_{i} X_{i} > 0.5m$
\end{claim}

\begin{proof}
Since we are considering median value to be our answer, if more than half the trials of A are correct, algorithm $A^{\ast}$ is also correct.
\end{proof}

\begin{claim}
To prove, $Pr[\Sigma_{i} X_{i} \ge 0.5m] \ge 1-\delta$ or $Pr[\Sigma_{i} X_{i} < 0.5m] < \delta$
\end{claim}

\begin{proof}
\begin{equation}
\begin{split}
Pr[\Sigma_{i} X_{i} < 0.5m] & = Pr[\Sigma_{i} X_{i} - 0.9m < -0.4m]\\
& \le Pr[|\Sigma_{i} X_{i} - \mu| > 0.4m]\\
& = Pr[|\Sigma X_{i} - \mu| > 0.4/0.9 \mu]
\end{split}
\end{equation}
Using Chernoff bound,
\begin{equation}
\begin{split}
& \leq e^{-c*0.9m}\\
& < \delta
\end{split}
\end{equation}
Above equation holds for $m = O(log(1/\delta))$
\end{proof}

\section{Distinct Elements}
Given, a stream of size $m$ containing numbers from $[n]$, we have to approximate the number of elements with non-zero frequency. To calculate the exact value the space required:

\begin{itemize}
\item $O(n)$ bits. (maintain a vector of length n).
\item $O(m \log (n))$ bits. (save m numbers, each taking $log(n)$ bits).
\end{itemize}

Since, this complexity is not feasible as $m$,$n$ can be very large, we'll look at algorithm for approximating the distinct count value.

\subsubsection{Hash Function}
\begin{itemize}
\item $h : [n] \rightarrow [0,1]$
\item $h(i)$ is uniformly distributed in $[0,1]$.
\end{itemize}

\subsection{Algorithm [Flajolet-Martin 1985]}
We maintain a variable $z$.
\begin{enumerate}
\item Initialize $z = 1$.
\item Whenever $i$ is encountered: $z = \min{(z,h(i))}$
\item When done, output $1/z -1$.
\end{enumerate}

Now, we'll prove the algorithm works in a similar fashion followed in previous lecture.
Let $d$ be number of distinct elements.

\begin{claim}
$E[z] = d+1$
\end{claim}

\begin{proof}
$z$ is the minimum of $d$ random numbers in $[0,1]$. Pick another random number $a \in [0,1]$. The probability $a<z$:
\begin{enumerate}
\item exactly z
\item probability it's smallest among $d+1$ reals : $1/(d+1)$
\end{enumerate}
Equating these two, one can prove the claim.
\end{proof}

\begin{claim}
$\text{var}[z] \leq 2/d^{2}$
\end{claim}

\begin{proof}
It can be done in a similar fashion described in previous lecture.
\end{proof}

\subsubsection{$(1+\eps)$ approximation Algorithm }
We can take $Z = (z_{1} + z_{2} + ... z_{k})/k$ for independent $z_{1}, ... z_{k}$

\subsection{Alternate Algorithm: Bottom-k}
Instead of just use the minimum value of hash function for $i$ inputs, we'll maintain the $k$ smallest hashes seen.
\begin{enumerate}
\item Initialize $(z_{1}, z_{2},...z_{k}) = 1$.
\item Keep $k$ smallest hashes seen, s.t. $z_{1}\leq z_{2}\leq...z_{k}$
\item When done, output $\hat{d} = k/z_{k}$
\end{enumerate}

\begin{claim}
The following claims are stated:
\begin{itemize}
\item $Pr[\hat{d} > (1 + \eps)d] \leq 0.05$
\item $Pr[\hat{d} < (1 - \eps)d] \leq 0.05$
\item Overall probability that $\hat{d}$ outside range is at most 0.1
\end{itemize}
\end{claim}

\begin{proof}
To compute $Pr[\hat{d} > (1+\eps)d]$:
\begin{itemize}
\item Define $X_{i} = 1$ iff $h(i) < \dfrac{k}{(1+\eps)d}$
\item Then $\hat{d} > (1+\eps)d$ iff $\Sigma_{i} X_{i} > k$
\item if $\Sigma_{i} X_{i} > k$\\
  $\iff \exists$ at least $k$ numbers for which $h(i) < \dfrac{k}{(1+\eps)d}$\\
    \begin{equation}
      \iff z_{k} < \dfrac{k}{(1+\eps)d}
      \iff \dfrac{k}{z_{k}} > (1+\eps)d
      \iff \hat{d} > (1+\eps)d
    \end{equation}
\item
  $E[X_{i}] = \dfrac{k}{(1+\eps)d}$\\
  $E[\Sigma_{i} X_{i}] = d E[X_{i}] = \dfrac{k}{1 + \eps}$\\
  $\text{var}[\Sigma_{i} X_{i}] = d \text{var}[X_{i}] \leq dE[X_{1}^{2}] \leq  \dfrac{k}{1+\eps} \leq k$\\
  (Since $X_{1} \in \{0,1\}$, $E[X_{1}^{2}] = E[X_{i}]$)
\item By Chebyshev:
    $Pr[|\Sigma X_{i} - \dfrac{k}{1+\eps}| > \sqrt{20k}] \leq 0.05 \implies Pr[\Sigma X_{i} > \dfrac{k}{1+\eps} + \sqrt{20k}] \leq 0.05 $\\
    \begin{itemize}
    \item
      (For $\eps < 1/2$ and $k=c/\eps^{2}$)\\
      $\dfrac{k}{1+\eps} + \sqrt{20k} \leq k(1-\eps+\eps^{2}) + \sqrt{20k}$ (Taylor Series Expansion)\\
      $ \leq k - k\eps/2 + 5\sqrt{c}/\eps$
      $ = k - c / 2\eps + 5\sqrt{c}/\eps$\\
      $ < k $ where $c > 100$
    \item
      Since $k > \dfrac{k}{1+\eps} + \sqrt{20k} $ in our case and $\Sigma X_{i}$ is monotonically increasing, $Pr[\Sigma X_{i} > k] \leq Pr[\Sigma X_{i} > \dfrac{k}{1+\eps} + \sqrt{20k}] \leq 0.05$

    \end{itemize}
\end{itemize}
\end{proof}

\subsection{Hash functions in stream}
The hash function we used has two practical issues: (1) the return value should be a real number. (2) how do we store it?

Discretization can solve the first issue. Instead of all the real numbers in $[0, 1]$, we use hash function with range $\{0, \frac{1}{M}, \frac{2}{M}, \frac{3}{M}, \ldots, 1\}$. For large $M \gg n^{3}$, the probability that $d \le n$ random numbers collide is at most $\frac{1}{n}$.

For the second issue, we use pairwise independent function instead of independent function.

\begin{definition}
$h: [n] \rightarrow \{1, 2, \ldots M\}$ is pairwise independent if for all $i \ne j$ and $a, b \in [M]$, $\text{Pr}[h(i)=a \land h(j)=b]=\frac{1}{M^2}$
\end{definition}

It works because in previous calculation, we only care about pairs. We defined $X_i=1$ iff $h(i)$ is small than a threshold, then we computed $\text{var}[\Sigma X_i] = E[(\Sigma X_i)^2] - E[(\Sigma X_i)^2] = E[X_1X_1 + X_1X_2 + \ldots]- E[(\Sigma X_i)^2]$. Notice that $E[X_iX_j]$ is the same for fully random $h$ and pairwise independent $h$.

\begin{example}
[Construct a pairwise independent hash]
Assume $M$ is a prime number (if not, we can always pick a larger $M$ that is a prime number). We pick $p, q \in \{0, 1, 2, \ldots M-1 \}$ and the hash function $h(i) = pi+q \mod M$. In this construction we only need $O(\log M) = O(\log n)$ space (to store $p, q, M$).
\end{example}

\begin{proof}
$h(i)=a, h(j)=b$ is equivalent to $pi+q \equiv a, pj+q \equiv b$. So $p(i-j) \equiv a-b$ and $p \equiv (a-b)(i-j)^{-1}, q \equiv a - pi$. Since $M$ is a prime number, the unique inverse implies that there is only one pair $(p, q)$ satisfies it. And the probability that pair is chosen is exactly $\frac{1}{M^2}$.
\end{proof}

\section{Impossibility Results}

We have used both approximation and randomization to solve the distinct counting problem with space much less than $\min{(m, n)}$. Now we are wondering: can we omit either approximation or randomization to achieve the same space efficiency? The answer is no.

\subsection{Deterministic Exact Won't Work}

First, we will show that there is no deterministic (no randomization) and exact (no approximation) way to solve it.

Suppose there do exists a deterministic and exact algorithm $A$ and an estimator function $R$ that use space $s \ll n, m$. That is, for a given integer stream, we first run the algorithm $A$ on the stream. As the stream goes $A$ will return middle memory steps, and we obtain the final memory state $\sigma$ after the stream ends. Then we apply $R$ on $\sigma$ to obtain our estimator $\hat{d}$. Since both $A$ and $R$ are deterministic and exact, $\hat{d}$ must equals to the distinct count for the stream.

We now build a binary representation $x$ of the stream with the following rules: (1) $x \in \{0, 1\}^{n}$, (2) $i$ in stream iff $x_i = 1$. For example, if 1, 3, 5, 6, 7 are in the stream and 2, 4 are not, $x$ will start with 1, 0, 1, 0, 1, 1, 1. Notice that each stream has a corresponding representation and streams containing different numbers have different representations.

\begin{claim}
We can recover the $x$ of the stream given the memory state $\sigma$
\end{claim}

\begin{proof}
Denote $d=R(\sigma)$ be the original estimator. Now we treat $\sigma$ as a middle snapshot of the memory and add integer $i$ as the next element of the stream. Now $A$ will return another memory state $\sigma'$, and $d'=R(\sigma)'$ will be our new estimator. If $d'=d$, $i$ must have appeared in the stream before since $A$ and $R$ are deterministic and exact. Similarly, if $d'>d$, $i$ must have not appeared in the stream before. Using this method with $i=1, 2, 3\ldots$ and we can recover the $x$.
\end{proof}

Since we can recover $x$ from $\sigma$, we can treat $\sigma$ as an encoding of a string $x$ of length $n$. But $\sigma$ has only $s \ll n$ bits! Furthermore, we can treat $A$, the function that produces $\sigma$, as a function with domain $\{0, 1\}^{n}$ and $\{0, 1\}^{s}$. We can see that $A$ must be injective because if $A(x)=A(x')=\sigma$, the recoverability implies $x=x'$.

Hence $s \ge n$. Which implies that there is no deterministic and exact algorithm $A$ and an estimator function $R$ that use space $s \ll n, m$.

\subsection{Deterministic Approx. Won't Either}

We can use the similar strategy to prove that deterministic approx. won't work. We pick $T \subset \{0, 1\}^{n}$ that satisfies the following conditions: (1) for all distinct $x, y \in T$, the number of digits $i$ that $y_i=1$ and $x_i=0$ should $\ge \frac{n}{6}$. (2) $|T| \ge 2^{\Omega(n)}$. Now we use algorithm $A$ to encode an input $x$ into $\sigma=A(x)$ and our estimator would be $\hat{d}=R(\sigma)$.

Now we want to recover $x$ based on $\sigma$, as what we have done in the last section. For a given $\sigma$ and any $y \in T$, we append $y$ to the stream and apply $A$ on it, and $A$ will return a memory state $\sigma'$. Using $\sigma'$ we have new estimator $\hat{d'}=R(\sigma')$.

\begin{claim}
If $\hat{d'} > 1.01 \hat{d}$, then $x \ne y$, else $x=y$.
\end{claim}

\begin{proof}
The idea is that when $x=y$, $\hat{d}$ would be really close to $\hat{d'}$ (up to $(1+\epsilon)^{2}$ because both of them are $\epsilon$-approximated) and when $x \ne y$, the construction of $T$ guarantee that $\hat{d} \ge \hat{d} + \frac{n}{6}$. So we can pick an $\epsilon$ that works for our claim.
\end{proof}

We can use this method to check every element $y \in T$ to see if $y=x$, and eventually we can recover $x$ from it. Similar to last section, we can show that $A$ is an injective function and it implies that $2^{s} \ge |T|$ or $s = \Omega(n)$.

\section{Concluding Remarks}

\begin{itemize}

\item We can use median trick and Chernoff bound to improve the probability of an existing algorithm.

\item For distinct elements problem, we can also store the hashes $h(i)$ approximately. One example is to store the number of leading zeros, and it only cost $O(\log \log n)$ bits per hash value, and that is the idea behind another algorithm called HyperLogLog.

\item For the impossibility results, we can also prove that randomized exact algorithm won't work.
\end{itemize}

\bibliographystyle{IEEEtran}
\bibliography{ref.bib}

\section*{Appendix}
\appendix
\section{K-means Algorithm Code in Python}
\begin{lstlisting}[language=Python]
import math
import matplotlib.pyplot as plt
import pandas as pd
import numpy as np


def loadData():
    df = pd.read_csv("./data/data.csv")
    return df.values


def euclideanDistance(vector1, vector2):
    return math.sqrt(sum(np.power(vector1 - vector2, 2)))


def initRandomCentroids(data, k):
    count, dim = data.shape
    centroids = np.zeros((k, dim))
    colMax = np.max(data, axis=0)
    colMin = np.min(data, axis=0)
    colRange = colMax - colMin
    for i in range(k):
        centroid = colMin + np.random.rand(dim) * colRange
        centroids[i, :] = centroid
    print(centroids)
    return centroids


def kmeans(k):
    data = loadData()
    count = data.shape[0]
    centroids = initRandomCentroids(data, k)
    clusterBound = np.zeros((count, 2))
    index = np.zeros((count, 1))
    processing = True
    while processing:
        processing = False
        for i in range(count):
            minIndex = 0
            minDist = float("inf")
            for j in range(k):
                distance = euclideanDistance(centroids[j, :], data[i, :])
                if distance < minDist:
                    minDist = distance
                    minIndex = j

            if clusterBound[i, 0] != minIndex:
                processing = True
                clusterBound[i, :] = minIndex, minDist ** 2
        index[:, 0] = clusterBound[:, 0]
        for j in range(k):
            newCentroid = data[np.all(index == j, axis=1), :]
            centroids[j, :] = np.mean(newCentroid, axis=0)
    print("k means finished!")
    visualization(centroids, clusterBound, data)


def visualization(centroids, clusterBound, data):
    plotMarkList = ['oy', 'og', 'or', 'oc', '^m', '+y', 'sk', 'dw', '<b', 'pg']
    centroidMarkList = ['Dr', 'Dc', 'Dm', 'Dy', '^k', '+w', 'sb', 'dg', '<r', 'pc']
    k = centroids.shape[0]
    count = data.shape[0]
    if data.shape[1] != 2:
        print("too many dimensions to draw :(")
        return
    if k > len(plotMarkList):
        print("too many centroids to draw :(")
        return
    for i in range(count):
        mark = plotMarkList[int(clusterBound[i, 0])]
        plt.plot(data[i, 0], data[i, 1], mark)
    for i in range(k):
        mark = centroidMarkList[i]
        plt.plot(centroids[i, 0], centroids[i, 1], mark)
    plt.show()


if __name__ == "__main__":
    kmeans(3)
\end{lstlisting}


\end{document}
